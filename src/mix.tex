\input cwebmac
% This file is part of MIX
% Copyright (C) 1998, 1999, 2000 Douglas Laing
% Copyright (C) 2000, 2005, 2007 Sergey Poznyakoff
%
% This program is free software; you can redistribute it and/or
% modify it under the terms of the GNU General Public License as
% published by the Free Software Foundation; either version 3, or (at
% your option) any later version.
%
% This program is distributed in the hope that it will be useful, but
% WITHOUT ANY WARRANTY; without even the implied warranty of
% MERCHANTABILITY or FITNESS FOR A PARTICULAR PURPOSE.  See the GNU
% General Public License for more details.
%
% You should have received a copy of the GNU General Public License
% along with this program.  If not, see <http://www.gnu.org/licenses/>.
%

\N{1}{1}Common definitions for MIX assembler and simulator.

The purpose is to simulate TAOCP MIX machine with 64 distinct
values per byte, 5 bytes per machine word and with address space
of 4000 words.

Internally, each byte will be represented by 6 bits, with a value
range of \PB{\T{\~00}} to \PB{\T{\~77}}. Sign is represented by a single bit, %
\PB{\T{0}} for
plus or \PB{\T{1}} for minus in the first position of the real word.

A word will then occupy 4 actual 8-bit hardware bytes of the real machine
and leave one bit (the second bit of the real word) over for possible
future use.

\Y\B\4\D$\.{MEMSIZE}$ \5
\T{4000}\par
\B\4\D$\.{BYTESIZE}$ \5
\T{6}\par
\B\4\D$\.{BYTESPERWORD}$ \5
\T{5}\par
\B\4\D$\.{B1}$ \5
$(\T{1}\LL\.{BYTESIZE}{}$)\par
\B\4\D$\.{B2}$ \5
$\.{B1}*{}$\.{B1}\par
\B\4\D$\.{B3}$ \5
$\.{B2}*{}$\.{B1}\par
\B\4\D$\.{B5}$ \5
$\.{B2}*{}$\.{B3}\par
\B\4\D$\.{MINUS}$ \5
((\&{unsigned}) \T{2}${}*\.{B5}{}$)\par
\B\4\D$\.{PLUS}$ \5
\T{0}\par
\B\4\D$\.{SIGN}(\|a)$ \5
$(\|a\AND(\.{MINUS}){}$)\C{ sign of word }\par
\B\4\D$\.{MAG}(\|a)$ \5
$(\|a\AND(\.{B5}-\T{1}){}$)\C{ magnitude of word }\par
\B\4\D$\|T$ \5
\T{1}\C{ True value }\par
\B\4\D$\|F$ \5
\T{0}\C{ False value }\par
\fi

\M{2}\PB{\&{struct} \\{mixop}} keeps data for a single MIX instruction. It is
used
by assembler (\PB{\\{mixal}}) to generate machine code and by terminal
emulator (\PB{$\\{mixsim}-\|t$}) to disassemble instructions.

\Y\B\&{struct} \&{mix\_op} ${}\{{}$\1\6
\&{char} ${}{*}\\{name}{}$;\C{ \PB{\.{OP}}: Option mnemonic name }\6
\&{int} \|f;\C{ \PB{(\|F)}: Default field notation }\6
\&{int} \|c;\C{ \PB{\|C}: Opcode }\6
\&{int} \\{notation};\C{ Field notation: 0 - (n), 1 - (l:r) }\2\6
${}\}{}$;\par
\fi

\M{3}External data provided by the automatically generated file optab.c
\Y\B\&{extern} \&{struct} \&{mix\_op} \\{mix\_optab}[\,];\C{ Array of available
\.{MIX} instructions }\6
\&{extern} \&{int} \\{op\_count};\C{ Number of entries in \PB{\\{mix\_optab}} }%
\6
\&{extern} \&{int} \\{dd\_cnt}[\,];\C{ Number of bytes in instruction. }\6
\&{extern} \&{int} \\{dd\_ent}[\,];\C{ Index to \PB{\\{mix\_optab}} }\6
\&{extern} \&{char} \\{mixalf}[\,];\C{ \.{MIX} character set }\par
\fi

\inx
\fin
\con
